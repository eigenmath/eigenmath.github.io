\documentclass[12pt]{article}
\usepackage[margin=2cm]{geometry}
\usepackage{amsmath}

\begin{document}

\noindent
The script ``schwarzchild-metric.txt'' shows that the
Einstein tensor $G_{\mu\nu}$ vanishes for the Schwarzchild metric
$$
g_{\mu\nu}=
\begin{pmatrix}
-(1-2M/r) & 0 & 0 & 0\\
0 & 1/(1-2M/r) & 0 & 0\\
0 & 0 & r^2 & 0\\
0 & 0 & 0 & r^2\sin^2\theta
\end{pmatrix}
$$

\end{document}

It turns out that the above $g_{\mu\nu}$ results in a $G_{\mu\nu}$ that is too complicated for Eigenmath to simplify so the following trick is used.
The metric tensor is defined as
$$
g_{\mu\nu}=
\begin{pmatrix}
-F(r) & 0 & 0 & 0\\
0 & 1/F(r) & 0 & 0\\
0 & 0 & r^2 & 0\\
0 & 0 & 0 & r^2\sin^2\theta
\end{pmatrix}
$$
where $F(r)$ is an unspecified function of $r$.
The result is
$$
G_{\mu\nu}=
\begin{pmatrix}
-\frac{FF'}{r}+\frac{F}{r^2}-\frac{F^2}{r^2} & 0 & 0 & 0
\\
0 & \frac{1}{r^2}+\frac{F'}{rF}-\frac{1}{r^2F} & 0 & 0
\\
0 & 0 & rF'+\frac{1}{2}r^2F'' & 0
\\
0 & 0 & 0 & rF'\sin^2\theta+\frac{1}{2}r^2F''\sin^2\theta
\end{pmatrix}
$$
After $G_{\mu\nu}$ is computed, $F(r)$ is defined as $1-2M/r$ and $G_{\mu\nu}=0$ is obtained.

\end{document}
