\documentclass[12pt]{article}
\usepackage{amsmath}
\usepackage{amssymb} % \mathbb
\usepackage{mathrsfs} % \mathscr
\usepackage[margin=2cm]{geometry}

\begin{document}

\noindent
{\bf Quantum electric field}

\bigskip
\noindent
Consider a light wave propagating in the $z$ direction.
For simplicity let the light be linearly polarized with electric
field vector $\bf E$ pointing in the $x$ direction.
\begin{equation*}
{\bf E}(t,x,y,z)=\begin{pmatrix}E_x\cos(kz-\omega t)\\0\\0\end{pmatrix}
\end{equation*}

\noindent
Symbol $\omega$ is angular frequency and $k$ is the wave number~$k=\omega/c$.

\bigskip
\noindent
The corresponding wave function is
\begin{equation*}
\psi=A\left|n-\tfrac{1}{2}\right\rangle
+B\left|n+\tfrac{1}{2}\right\rangle
\end{equation*}

\noindent
where $n$ is the number of photons per unit volume and
\begin{align*}
A&=\exp\left(-i\left(n-\tfrac{1}{2}\right)\omega t\right)\\
B&=\exp\left(-i\left(n+\tfrac{1}{2}\right)\omega t\right)
\end{align*}

\noindent
The electric field operator is
\begin{equation*}
\hat{\mathscr{E}}=C\hat{a}+C^*\hat{a}^\dagger
\end{equation*}

\noindent
where $\hat{a}$ and $\hat{a}^\dagger$ are the lowering and raising operators such that
\begin{align*}
a\left|n+\tfrac{1}{2}\right\rangle&=\sqrt{n}\,\left|n-\tfrac{1}{2}\right\rangle\\
a^\dagger\left|n-\tfrac{1}{2}\right\rangle&=\sqrt{n}\,\left|n+\tfrac{1}{2}\right\rangle
\end{align*}

\noindent
The quantity $C$ is
\begin{equation*}
C=\sqrt{\frac{\hbar\omega}{2V\varepsilon_0}}\,\exp(ikz)
\end{equation*}

\noindent
where $V$ is a unit volume.

\bigskip
\noindent
Apply electric field operator $\hat{\mathscr{E}}$ to wave function $\psi$.
\begin{align*}
\hat{\mathscr{E}}\psi
&=C\hat{a}\psi+C^*\hat{a}^\dagger\psi\\
&=CA\sqrt{n-1}\,\left|n-\tfrac{3}{2}\right\rangle+CB\sqrt{n}\,\left|n-\tfrac{1}{2}\right\rangle
+C^*A\sqrt{n}\,\left|n+\tfrac{1}{2}\right\rangle+C^*B\sqrt{n+1}\,\left|n+\tfrac{3}{2}\right\rangle
\end{align*}

\noindent
The observed electric field is the eigenvalue $\mathscr{E}$ such that $\hat{\mathscr{E}}\psi=\mathscr{E}\psi$.
\begin{align*}
\mathscr{E}
&=\psi^\dagger\hat{\mathscr{E}}\psi\\
&=\left\langle n-\tfrac{1}{2}\right|A^*CB\sqrt{n}\,\left|n-\tfrac{1}{2}\right\rangle
+\left\langle n+\tfrac{1}{2}\right|B^*C^*A\sqrt{n}\,\left|n+\tfrac{1}{2}\right\rangle\\
&=\sqrt{n}\,C\exp(-i\omega t)+\sqrt{n}\,C^*\exp(i\omega t)\\
&=\sqrt{\frac{2n\hbar\omega}{V\varepsilon_0}}\cos(kz-\omega t)
\end{align*}

\noindent
Identifying $\mathscr{E}$ as the first component of $\mathbf{E}$ we have $\mathscr{E}=E_x\cos(kz-\omega t)$.
Hence the electric field amplitude $E_x$ is proportional to the square root of photon density.
$$
E_x=\sqrt{\frac{2n\hbar\omega}{V\varepsilon_0}}
$$

\noindent
Run ``quantum-electric-field-1.txt'' to verify.

\bigskip
\noindent
The SI unit of electric field strength is volts per meter.
The script ``quantum-electric-field-2.txt'' calculates the SI constant
for converting photon density to volts per meter.
Yellow light with wavelength $\lambda=600$ nanometers is used for
angular frequency~$\omega$.
The result is
$$
\sqrt{\frac{2n\hbar\omega}{V\varepsilon_0}}
=2.7\times10^{-4}\,\text{volt}\,\text{meter}^{-1}\times\sqrt{n}
$$

\noindent
The symbol $V$ is a one cubic meter unit volume.
The script also converts volts per meter to base units.
$$
1\,\text{volt}\,\text{meter}^{-1}
=1\,\text{kilogram}\,\text{meter}\,\text{ampere}^{-1}\,\text{second}^{-3}
$$

\bigskip
\noindent
{\bf Reference}

\bigskip
\noindent
Dommelen, Leon van. ``Quantum Mechanics for Engineers, Section A.23 Quantization of radiation.''
\verb$http://www.eng.fsu.edu/~dommelen/quantum/style_a/qftqem.html$

\end{document}
